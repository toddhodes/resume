\documentstyle[palatino,url]{article}
\setlength{\oddsidemargin}{0.25 in}
\setlength{\evensidemargin}{-0.25 in}
%%\setlength{\topmargin}{-0.6 in}
\setlength{\topmargin}{-0.95 in}
\setlength{\textwidth}{6.5 in}
%%\setlength{\textheight}{8.5 in}
%%%\setlength{\textheight}{9 in}
\setlength{\textheight}{10 in}
%%\setlength{\headsep}{0.55 in}
\setlength{\headsep}{0.35 in}
\setlength{\parindent}{0 in}
%%\setlength{\parskip}{0.1 in}
\setlength{\parskip}{0.01 in}


\addtolength{\evensidemargin}{-15pt}
\addtolength{\oddsidemargin}{-15pt}

\pagestyle{empty}

\begin{document}
\begin{center} \bf
\Large
                               Todd D. Hodes
\medskip
\normalsize


\begin{tabular}{lp{3.75in}r}
6709 Arlington Blvd.  & & todd@toddh.org \\
Richmond, CA  94805 & & \url{http://toddh.org} \\
(510)  332-0399 \\
\end{tabular}

\end{center}

\bigskip


\begin{bf} \large
Education \\[-18pt]
\end{bf}

\begin{tabbing}
xxi \= \kill
\>   Ph.\ D. in Computer Science, 2002 \\
\>   M.\ S. in  Computer Science, 1997 \\
\>   University of California, Berkeley \\
\smallskip \\[-6pt]
\>   B.\ S. in Computer Science, high honors, 1994 \\
\>   University of Virginia \\
\end{tabbing}

\begin{bf} \large
Honors, Awards, Achievements  \\[-16pt]
\end{bf}

\begin{itemize}
\item 2007 Webby Award (Mobile) \& Webby People’s Choice Award (Mobile) for m.ask.com \\[-18pt]
\item  Best Paper Award, 3rd ACM/IEEE International Conference on Mobile Computing, 1997 \\[-18pt]
\item  California Fellowship in Microelectronics,
        University of California, Berkeley, 1994-1995 \\[-18pt]
\item  Louis T. Radar Award, U. of Virginia Computer Science Department, April 1994 \\[-18pt]
\item  Dean's List, Intermediate Honors, Tau Beta Pi, Golden Key, University of Virginia 1991-1995 \\[-18pt]
\item  National Merit Scholar, 1990 \\[-18pt]
\item  American Computer Science League All-Star Contest, 
       first place in division, Washington, DC, 1990 \\[-18pt]
\item    Duke University Summer Computing Program, Full Scholarship, 1987 \\
\end{itemize}


%%%%%% \newpage

\begin{bf} \large
Publications
\end{bf}
\begin{tabbing}
xxi \= xx \= x \= \kill

\>    Jesse Meyers, Scott Hotes, Todd Hodes \\
\>\>      ``System and method for range search over distributive storage systems'' \\
\>\>    Patent \#8,924,365, issued 12/2014. \\

\smallskip \\[-3pt]
\>    Brian Martin, Joe Augst, Jesse Meyers, Todd Hodes, Scott Hotes \\
\>\>      ``System and method for managing third party application program access to user information \\
\>\>      via a native application program interface'' \\
\>\>    Patent \#8,924,365,    issued 3/2014. \\

\smallskip \\[-3pt]
\>    Adrian Freed, Todd Hodes, John Hauser \\
\>\>      ``Apparatus and Method of Additive Synthesis of Digital Audio Signals \\
\>\>   Using a Recursive Digital Oscillator'' \\
\>\>    Patent \#7,317,958: UC Berkeley Regents, Berkeley, California, issued 1/2008. \\

\smallskip \\[-3pt]
\>    Todd Hodes \\
\>\>      ``Discovery and Adaptation for Location-Based Services'' \\
\>\>    Ph.D. thesis, University of Califonia, Berkeley. 2002. \\
\>\>    Randy Katz, thesis advisor \\

\smallskip \\[-3pt]
\>    T. Hodes, S. Czerwinski, B. Zhao, A. Joseph, R. H. Katz \\
\>\>      ``An Architecture for Secure Wide-area Service Discovery'' \\
\>\>       ACM Wireless Networks Journal, Special Issue \\
\>\>       Volume 8, Issue 2/3, March/May 2002, pp. 213-230 \\

\smallskip \\[-3pt]
\>   M. Munson, T. Hodes, T. Fischer, K. H. Lee, T. Lehman, B. Zhao \\
\>\>      ``Flexible Internetworking of Devices and Controls'' \\
\>\>        25th Annual Conference of the IEEE Industrial Electronics Society (IECON),  \\
\>\>	    San Jose, CA, December 1999. \\

\end{tabbing}
\newpage
\begin{bf} \large
Publications, cont.
\end{bf}
\begin{tabbing}
xxi \= xx \= x \= \kill

\smallskip \\[-3pt]
\>    T. D. Hodes, R. H. Katz \\
\>\>      ``Composable Ad hoc Location-based Services for Heterogeneous
             Mobile Clients,'' \\
\>\>       ACM Wireless Networks Journal, Special issue on Mobile Computing \\
\>\>       Vol. 5, No. 5, October 1999, pp. 411-427. \\
\smallskip \\[-3pt]
\>    T. Hodes, R. H. Katz \\
\>\>      ``A Document-based Framework for Internet Application Control'' \\
\>\>       2nd USENIX Symposium on Internet Technologies and Systems \\
\>\>       Boulder, CO, October 1999, pp. 59-70. \\
\smallskip \\[-3pt]
\>    T. Hodes, J. Hauser, A. Freed, J. Wawrzynek\\
\>\>      ``Second-order Recursive Oscillators for Musical Additive Synthesis \\
\>\>\>        Applications on SIMD and VLIW Processors,'' \\
\>\>       International Computer Music Conference (ICMC), Beijing, China, October 1999. \\
\smallskip \\[-3pt]
\>    S. Czerwinski, B. Zhao, T. Hodes, A. Joseph, R. H. Katz \\
\>\>      ``An Architecture for a Secure Service Discovery Service'' \\
\>\>       5th ACM/IEEE International Conference on Mobile Computing \\
\>\>       Seattle, WA, August 1999, pp. 24-35. \\
\smallskip \\[-3pt]
\>    Steven McCanne, Eric Brewer, Randy Katz, Elan Amir, Yatin Chawathe,
    Todd Hodes, et. al. \\
\>\>      ``MASH: Enabling Scalable Multipoint Collaboration'' \\
\>\>       ACM Computing Surveys, Volume 31, No. 2es, June 1999. \\
\smallskip \\[-3pt]
\>    T. Hodes, J. Hauser, A. Freed, J. Wawrzynek, D. Wessel \\
\>\>      ``A Fixed-point Recursive Digital Oscillator for Additive Synthesis
        of Audio,'' \\
\>\>       IEEE International Conference on Acoustics, Speech, and Signal
            Processing,  \\
\>\>        Phoenix, Arizona, March 1999. \\
\smallskip \\[-3pt]
\>    T. Hodes, M. Newman, S. McCanne, R. H. Katz, J. Landay \\
\>\>      ``Shared Remote Control of a Videoconferencing Application: 
            Motivation, Design, and Implementation,'' \\
\>\>       SPIE Multimedia Computing and Networking 1999,  \\
\>\>        San Jose, California, January 1999, pp. 17-28. \\
\smallskip \\[-3pt]
\>    E. Brewer, R. H. Katz, E. Amir, H. Balakrishnan, Y. Chawathe, A. Fox,
    S. Gribble, T. Hodes, G. Nguyen, \\
\>\>\> V. Padmanabhan, M. Stemm, S. Seshan, T. Henderson \\
\>\>      ``A Network Architecture for Heterogeneous Mobile Computing,'' \\
\>\>       IEEE Personal Communications Magazine, October 1998. Invited Paper. \\
\smallskip \\[-3pt]
\>    T. D. Hodes, R. H. Katz \\
\>\>      ``Enabling `Smart Spaces:' Entity Description and User Interface
        Generation \\
\>\>\>     for a Heterogeneous Component-Based Distributed System,'' \\
\>\>       DARPA/NIST Smart Spaces Workshop,  \\
\>\>       Gaithersburg, Maryland, July 1998.  pp. 7/44-7/51.  \\
\>\>       also, UC Berkeley Technical Report CSD/98/1008. \\
\smallskip \\[-3pt]
\>    T. D. Hodes, R. H. Katz, E. Servan-Schreiber, L. A. Rowe \\
\>\>      ``Composable Ad hoc Mobile Services for Universal Interaction,'' \\
\>\>       3rd ACM/IEEE International Conference on Mobile Computing,  \\
\>\>        Budapest, Hungary, September 1997, pp. 1-12. \\
\>\>        {\em Best Paper award. }\\
\smallskip \\[-3pt]
\>    T. D. Hodes \\
\>\>      ``Recursive Oscillators on a Fixed-Point Vector Microprocessor \\
\>\>\>     for High Performance Additive Synthesis of Audio,'' \\
\>\>       MS report, December 1997. also, UCB Technical Report CSD/98/1007. \\

\end{tabbing}
\newpage
\begin{bf} \large
Publications, cont.
\end{bf}
\begin{tabbing}
xxi \= xx \= x \= \kill

\smallskip \\[-3pt]
\>    R. H. Katz, E. A. Brewer, E. Amir, H. Balakrishnan, A. Fox,
       S. Gribble, T. Hodes, \\
\>\>\>   D. Jiang, G. Nguyen, V. Padmanabhan, M. Stemm. \\
\>\> ``The Bay Area Research Wireless Access Network (BARWAN),''  \\
\>\> 41st IEEE Computer Society International Conference (COMPCON), 1996. \\
\smallskip \\
\>	W. T. Fennell, Jr., T. Hodes, S. Witherell, C. Goebel, 
	R. Thakkar, T. Schwenk, \\
\>\>	``Method of Managing Multi-Player Game Playing Over a Network,''  \\
\>\>	Patent \#5,695,400: BoxerJAM Films, Charlottesville, Virginia 
(filed 1/96, accepted 12/97).  \\
\smallskip \\
\>    T. D. Hodes, B. A. McCoy, G. Robins \\
\>\>  ``Dynamically Wiresized Elmore-Based Routing Constructions,''  \\
\>\>   1994 IEEE International Symposium on Circuits and Systems, \\
\>\>   London, England, May 1994, Volume I, pp. 463-466. \\
\smallskip \\
\end{tabbing}

%\newpage
%\begin{bf} \large
%Publications, cont.
%%\end{bf}
%\begin{tabbing}
%xxi \= xx \= x \= \kill
%\end{tabbing}

%%\bigskip

\begin{bf} \large
Work Experience
\end{bf}
\medskip

\begin{tabular}{p{5.2in}@{\hspace{1.5cm}}l}

	\em Senior Staff Engineer, Director of Engineering, Senior Scientist 	    & June 2002 -  \\
	 \em  Avast s.r.o. / AVG LLC / LocationLabs LLC (dba WaveMarket)  	& January 2021 \\
	
	   We white-label family safety applications for Cellular telephone carriers.	

	   Took series-A startup from no products and no income to \$220,000,000
	   acquisition by AVG LLC, wearing various engineering hats.  
	   AVG was then itself acquired by Avast corp, 
	   where I continued on the engineering team as Senior Staff Engineer.
	   Original title in 2002 was Senior Scientist.  In 2006 became Director of Engineering and Director of
	   Mobile Technologies. In 20015 became Senior Staff Engineer.
	   Started as full-stack backend and web, then specialized in J2ME, currently specialized in Android.

	   Various projects I committed to are highlighted in the \textbf{Projects} section. \\[9pt] 	  
	  
    \em Graduate Student Researcher, UC Berkeley Computer Science division
                                                                 & 1995-2002 \\
       Work on mobile computing \& location-based applications 
       [WINET99, USITS99, PersComm98]; 
       multimedia networking [CompSurveys99, MMCN99];
       peer-to-peer service location [Mobicom99];
       fast sine synthesis techniques with vector instruction 
       sets [ICASSP99] \\[9pt]

    \em Technical Staff, Luxxon corporation, San Jose, California. & 2000-2001 \\
       Worked one day a week assisting with R\&D, especially
       protocols for discovery/negotiation of client
       device characteristics and media caching \\[9pt]

    \em Graduate Internship and Contracting, IBM Almaden Research Center, 
          Almaden, California
                                                              & 1999-2000 \\
       Summer internship and follow-up continuing work with
       the TSpaces middleware project [IECON99] \\[9pt]

    \em Graduate Student Instructor, UC Berkeley Computer Science division
                                                                 & Fall 1996 \\
       Teaching Assistant for {\em Computer Architecture and
       Engineering} (CS152), taught by David Patterson and
       Robert Yung \\[9pt]

    \em BoxerJam Films, Charlottesville, Virginia                & 1994 \\
       Assisted the early-stage design and implementation of 
       a multi-platform
       client/server system communicating via both modems and IP
       networks.  Addressed wide-area latency-hiding and user interface
       issues.  The resulting game, "Strike-A-Match,"
       was deloyed to America Online (AOL) and Yahoo! games 
       [Patent \#5,695,400] \\[9pt]

    \em Unix Consultant, U. of Virginia Information, 
         Technology, and Communications  (ITC)              & 1993-1994 \\
       Answered questions via phone dealing with issues on all the
       University's available UNIX platforms, including SunOS,
       AIX, IRIX, and NeXT \\[9pt]

    \em HBO \& Company, Advanced Technologies Group, 
           Atlanta, Georgia                                      & 1991,1993 \\
       Designed and implemented interfaces and utilities to
       assist integrating PC-based client machines with their
       legacy mainframe (MV/40000) medical system. \\[9pt]

\end{tabular}


\bigskip


\newpage
\begin{bf} \large
Projects:
\end{bf}
\medskip

Current deployed software: 
	   	   
	   \begin{itemize}
		\item Verizon Smart Family: Android architecture \& client team \\[-16pt]
		\item AT\&T Secure Family: Android architecture \& client team \\[-16pt]
		\item T-Mobile FamlyWhere: Android architecture \& client team \\[-16pt]
		\item Sprint Familly Locator (SFL): Android architecture \& client team  \\[-16pt]
	\end{itemize}

Previous projects: 

	   \begin{itemize}
		\item AT\&T FamilyMap: Android architecture \& client team and Windows Phone 7 sole contributor  \\[-16pt]
		\item Verizon Family Base: Android architecture \& client team  \\[-16pt]

		\item \url{veriplace.com} Sparkle location-as-a-service: A suite of thin smartphone application agents providing end-to- end user-plane locates (hybrid cell-sector/WiFi/GPS), avoiding carrier infrastructure for cost reasons. API/architecture, sole contributor to Windows Mobile implementation, contributor to RIM/Android shared codebase.

		\item \url{streethive.com}: location-based social sharing site, focused on situating posts (photos/comments) atop a map. J2ME handset client, web.  and shared responsibility for architecting the backend/APIs so that they were amenable to both web and mobile content access.

		\item \url{m.ask.com}: xHTML-MP/WAP portal access to ask.com syndication backend.Web search, images search, maps/directions, weather, horoscopes, etc. (myself and one other engineer)


		\item \url{cmprssr.com}: In order to view the resulting content discovered via queries on m.ask.com, web pages needed to be transcoded from free-form HTML to valid xHTML-MP. cmprssr provided this service at extremely high volume (millions/day). (team of myself and one other engineer)

  		\item InterActiveCorp / IAC \url{gps.ask.com}: voice turn-by-turn navigation on GPS-enabled feature phones to aggregated IAC properties CitySearch, Evite, Ticketmaster, Ask.com. The underlying technology in- novation used to manage app complexity under memory and processing constraints was the Dy- namic Mapping Environment (DyME) toolset. DyME provides XML-defined UI specification (ala Silverlight’s XAML), but also, XML-defined screen and action flows called “chains”. These XML de- scriptions were compiled down into primitives for efficiency, and could be updated at runtime, to overcome the lack of a ClassLoader in Java ME (pre-MIDP3.0). (team of five engineers plus support)
	   
  		\item MapMe: on-handset point-of-interest finders + route generator, communicating to a custom geoserver backend. WaveMarket’s first handset software product releases. Responsible for hardware acquisition and co-location hosting, OS installation and configuration, sysadmin / operations, Cisco VPN to partner site, hardware load-balancing and failover (via Alteon ACEdirector), data integration, plus all the client and server application software itself. Code size was limited to 100KB, heap to 256KB.
		
	\end{itemize}



\end{document}



