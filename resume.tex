\documentstyle[palatino,url]{article}
\setlength{\oddsidemargin}{0.25 in}
\setlength{\evensidemargin}{-0.25 in}
%%\setlength{\topmargin}{-0.6 in}
\setlength{\topmargin}{-0.95 in}
\setlength{\textwidth}{6.5 in}
\setlength{\textheight}{9.6 in}
%%\setlength{\headsep}{0.55 in}
\setlength{\headsep}{0.35 in}
\setlength{\parindent}{0 in}
%%\setlength{\parskip}{0.1 in}
\setlength{\parskip}{0.05 in}


\addtolength{\evensidemargin}{-15pt}
\addtolength{\oddsidemargin}{-15pt}

\pagestyle{empty}

\begin{document}
\begin{center} \bf
\Large
                               Todd D. Hodes
\medskip
\normalsize


\begin{tabular}{lp{3.0in}r}
Richmond, CA 94805    & & todd@toddh.org \\
(510)  332-0399& & \url{toddhodes.com} \\
\end{tabular}

\end{center}

\bigskip


\begin{bf} \large
Education \\[-18pt]
\end{bf}

\begin{tabbing}
xxi \= \kill
\>   Ph.D. in Computer Science (Systems/Networking), 2002  \\
\>   M.S. in Computer Science (Computer Music), 1997  \\
\>   University of California, Berkeley  \\
\smallskip \\[-6pt]
\>   B.S. in Computer Science, 1994 \\
\>   University of Virginia, Charlottesville \\
\end{tabbing}


\begin{bf} \large
Technical experience  \\[-16pt]
\end{bf}

\begin{itemize}
\item Programming languages: Kotlin, Java, shell, SQL, C/C++, C\#, javascript, python. \\[-18pt]
\item Build tools: git, TeamCity, Jenkins, Tinderbox.   \\[-18pt]
\item Collaboration tools: Jira, Confluence, Figma, Miro, Zeplin, Bugzilla.    \\[-18pt]
\item Cloud \& containers: EC2, S3, ELB, EMR, docker.  \\[-18pt]
\item Android libraries:
	   RxJava2, Mosby MVP, Jetpack, Material Design,
	   Amplitude, Adjust, Conductor, Dagger, Datadog, Branch.io, Cloudinary, Dagger, Epoxy, Lottie, EventBus, Crashlytics,
	   Google Maps, MapQuest/MapBox, Gson, Guava, Joda, OkHttp/Retrofit, Optimizely, Picasso, PubNub,
	   Realm, Room, Loggly. \\
\end{itemize}

%%%%%%%%%%%%%%%%%%%%%%%%%%%%%%%%

\begin{bf} \large
Work Experience
\end{bf}
\medskip

\begin{tabular}{p{5.2in}@{\hspace{1.5cm}}l}

	\bf  Avast s.r.o., Emeryville, California   	    & July 2016 -  \\
	\bf  Senior Staff Engineer	& January 2021 \\[4pt]
	
	   Avast acquired AVG in 2016 and took them private.  My title changed to Senior Staff Engineer,
	   reporting to the VP of Engineering. 
	   I continued working on the white-label carrier family safety applications --
           please refer to the entries below and {\it Projects} section for details.
	   Provided programming \& architecture, 
		feature ownership, build support, tooling, infrastructure support. 
	    In addition to remaining as a cross-team architect and tech lead,
	   I restarted our bi-weekly developer demo and showcase series for the SF Bay offices.
		 \\[9pt]
	   
	\bf AVG Technologies LLC, Emeryville, California   	& October 2014 -  \\
	\bf  Director of Engineering (Mobile)  \& Software Architect 	& July 2016 \\[4pt]
	
	   AVG acquired LocationLabs in 2014. We spent the next two years hitting our contracted
	   financial targets to maximize the payout (\$220MM).   
	   I continued to focus on the Android versions of the white-label carrier family safety applications.
	   Please refer to the entry below, and {\it Projects} section, for details. \\[9pt]
	   
	
	   

\end{tabular}

\newpage
\begin{bf} \large
Work Experience, cont.
\end{bf}
\medskip

\begin{tabular}{p{5.2in}@{\hspace{1.5cm}}l}
	  
	\bf  LocationLabs LLC (dba WaveMarket, Inc.), Emeryville, California    & June 2002 -  \\
	\bf  Senior Scientist, Senior Software Engineer, Director of Engineering (Mobile) & October 2014 \\[4pt]
	
	   I was one of the first dozen employees at Berkeley-founded, and then Emeryville-based, 
	   WaveMarket/LocationLabs,
	   an early-stage startup that received its series A in 2001.
	   We went from no deployed products in 2002 to an
	   acquisition by AVG Technologies in 2014.  I wore every engineering hat, starting with responsibility
	   for full-stack client, server, and SysOps to
	   then specializing in mobile development as we grew.  
	   We started small -- 
	   I drove my car over to install the servers, load balancer, and VPN in our first colo space in Oakland.s
	   There were many product pivots along the way. Please refer to the {\it Projects} section for a sampling, 
	   and, I am happy to describe any of them in much more detail on request.
	   At first, these were written for J2ME/BlackBerry, PocketPC, and mobile web (xHTML).
	   Once iOS and Android unseated handsets, 
	   I specialized in Android, and, client/server architectural co-design.  In addition to mobile development, I was an acting
	   Partner Engineer, visiting carriers after we had signed contracts
	   with them and were implementing APIs/schema design for the integration points
	    (authentication, account management, payment, location data access, unmetered data access, messaging). \\[9pt]
       	  
    \bf Luxxon Corporation, San Jose, California. & 2000-2001 \\[3pt]
       R\&D of their multimedia transcoding proxy
       protocols for discovery and negotiation of client
       device characteristics and media caching. \\[9pt]

    \bf IBM Almaden Research Center, 
          Almaden, California                         & Summer 1999 \\[3pt]
       Graduate Internship and follow-up work with
       the IBM Almaden T-Spaces middleware project. \\[9pt]

    \bf BoXerJam, 
         Charlottesville, Virginia                        & Summer 1994 \\[3pt]
       Film company turned software startup. 
       Wrote client and server code for Strike-a-Match, deployed to AOL and Yahoo! Games \\[9pt]

\end{tabular}


%%%%%%%%%%%%%%%%%%%%%%%%%%%%

%\newpage
\medskip
\bigskip


\begin{bf} \large
Projects
\end{bf}
	   	   
	   \begin{itemize}
		\item 
		{\em Verizon Smart Family,
		AT\&T Secure Family, 
		T-Mobile FamilyWhere,
		Sprint Family Locator,
		Verizon FamilyBase,
		AT\&T FamilyMap} Android apps: 
		
		Ongoing contributor (implementation, maintenance) to a 1MM+ LOC Android codebase that comprised
		various white-label Parental Controls applications. There were two major feature 
		categories.  {\it Controls} which collected data about device usage from a ``child'' phone and displayed it
		in rich UIs on a ``parent'' phone, and, {\it Location}, map-based features such as geofences, location polling, 
		location check-in, and ride requests.
		
		One specific focus area was the implementation and maintenance of the child-phone data extraction -- 
		call, message, \& contact list collection, 
		network traffic tracking, and
		per-application usage (foreground activity) timing.  
		Another focus area was implementing and maintaining the {\it Content Filtering} library.  
		This was anchored by an a on-device VPN service that intercepted only
		DNS traffic (DNS over TLS man-in-the-middle, or unencrypted port 53), without re-routing application traffic.
		A ``web shield'' blocking feature leveraged
		this control over DNS mappings to support both web blocking and app network traffic blocking.  Additionally,
		there was an Accessibility (a11y) service extension that we used to 
		\begin{itemize}
		\item monitor activities to provide app UI auto-minimize, including floating players in PIP,
		\item detect \& disable attempts to uninstall the app, and
		\item automate setup -- enable Device Admin, VPN, and related Settings toggles.
		\end{itemize}
		
		This project went through a major refactor to MVP + Rx when a pair of separated Controls and Location
		apps were combined, updating an aging codebase to best practices.  
		
		In addition to providing software engineering over the lifetime of the app, 
		I migrated to technical leadership in the director/staff role, dealing more with
		programming architecture, 
		feature ownership (Location and Controls horizontals), API design,
		build process, testing/debug support, and related tooling \& scripting.  
		\\[-16pt]
		
		\item {\em  AT\&T FamilyMap} WindowsPhone app: I wrote our WinPhone 7+
			implementation of the AT\&T family locator. Our partners at AT\&T requested that 
			we support it natively for their new phones with the new OS, and pre-load it. 
			Worked at the Microsoft campus in Redmond under NDA
			finalizing the builds.
			It ended up preloaded on every Windows phone sold by AT\&T. \\[-16pt]

		\item \url{veriplace.com}: Carrier and non-carrier cell phone location as a API service. 
		A suite of thin smartphone application agents providing
		end-to-end user-plane locates (hybrid cell-sector/WiFi/GPS), with the option of using the 
		carrier networks (Verizon, AT\&T, Sprint) LPS/MLP/GMLC, or, avoiding carrier infrastructure for cost reasons.
		API/architecture, sole contributor to Windows Mobile implementation, contributor to Android/J2ME/RIM
		shared codebase.

		\item \url{streethive.com}: An early location-based social sharing site: 
		StreetHive and Crunkie were social networks 
		focused on users sharing photos and text/comments atop a map. Supported on 
		handset clients (J2ME, RIM, xHTML, WAP) and web.  
		I worked on the handset client team, and shared responsibility for architecting the backend/APIs such that
		they were amenable to both web and 
		mobile content access. Java/J2ME front-end, Java/MySQL backend.

		\item \url{m.ask.com}: One of the worlds first mobile portals.  xHTML-MP/WAP access to the ask.com
		syndication backend. Provided access to full web search, 
		image search, maps with turn-by-turn directions, weather, horoscopes, and more. 
		Implemented with Java servlets/JSP + MySQL 
		by myself and one other engineer.  {\em The site won two Webby Awards in the Mobile category 
		(both Jury and People's Choice).}

		\item \url{cmprssr.com}: In order to view the resulting content discovered via queries on m.ask.com
		on feature-poor handsets, web 
		pages needed to be transcoded from free-form HTML to valid xHTML-MP. {\it cmprssr} provided this service at
		extremely high volume (millions/day). Written in perl by myself and one other engineer.

  		\item IAC \url{gps.ask.com}: voice turn-by-turn navigation on GPS-enabled feature phones
		 to aggregated InterAactiveCorp (IAC) properties CitySearch, Evite, Ticketmaster, Ask.com. 
		 The underlying technology innovation
		  used to manage app complexity under memory and processing constraints was a declarative UI screen specification
		   augmented with novel XML-defined action flows called ``chains''. These XML descriptions were compiled down
		    into primitives for efficiency, and could be updated at runtime, to overcome the lack of a ClassLoader 
		    in JavaME. 
	   
  		\item {\em MapMe} \& {\em NearHere}:
		Early on-handset point-of-interest finders + route generator, communicating to a custom geoserver
		backend. WaveMarket's first handset software product releases. Responsible fo co-location hosting,
		OS installation and configuration, sysadmin / operations, Cisco VPN to partner site, hardware load-balancing
		and failover (via Alteon ACEdirector), data integration, plus all the client and server application software itself.
		This on very limited hardware -- code size was limited to 100KB, heap to 256KB.
		
	\end{itemize}

%%%%%%%%%%%%%%%%%%%%%%%%

\bigskip
\bigskip

\begin{bf} \large
Patents \& Publications
\end{bf}
\begin{tabbing}
xxi \= xx \= x \= \kill

\>    Jesse Meyers, Scott Hotes, Todd Hodes \\
\>\>      ``System and method for range search over distributive storage systems'' \\
\>\>    Patent \#8924365, issued 12/2014. \\

\smallskip \\[-3pt]
\>    Brian Martin, Joseph Augst, Jesse Meyers, Todd Hodes, Scott Hotes \\
\>\>      ``System and method for managing third party application program access to user information \\
\>\>      via a native application program interface'' \\
\>\>    Patent \#8683554, issued 3/2014. \\

\end{tabbing}
\newpage
\begin{bf} \large
Patents \& Publications, cont.
\end{bf}
\begin{tabbing}
xxi \= xx \= x \= \kill

\smallskip \\[-3pt]
\>    Adrian Freed, Todd Hodes, John Hauser \\
\>\>      ``Apparatus and Method of Additive Synthesis of Digital Audio Signals \\
\>\>   Using a Recursive Digital Oscillator'' \\
\>\>    Patent \#7317958: UC Berkeley Regents, Berkeley, California, issued 1/2008. \\

\smallskip \\[-3pt]
\>    Todd Hodes \\
\>\>      ``Discovery and Adaptation for Location-Based Services'' \\
\>\>    Ph.D. thesis, University of Califonia, Berkeley. 2002. \\
\>\>    Randy Katz, thesis advisor \\

\smallskip \\[-3pt]
\>    T. Hodes, S. Czerwinski, B. Zhao, A. Joseph, R. H. Katz \\
\>\>      ``An Architecture for Secure Wide-area Service Discovery'' \\
\>\>       ACM Wireless Networks Journal, Special Issue \\
\>\>       Volume 8, Issue 2/3, March/May 2002, pp. 213-230 \\

\smallskip \\[-3pt]
\>   M. Munson, T. Hodes, T. Fischer, K. H. Lee, T. Lehman, B. Zhao \\
\>\>      ``Flexible Internetworking of Devices and Controls'' \\
\>\>        25th Annual Conference of the IEEE Industrial Electronics Society (IECON),  \\
\>\>	    San Jose, CA, December 1999. \\

\smallskip \\[-3pt]
\>    T. D. Hodes, R. H. Katz \\
\>\>      ``Composable Ad hoc Location-based Services for Heterogeneous
             Mobile Clients,'' \\
\>\>       ACM Wireless Networks Journal, Special issue on Mobile Computing \\
\>\>       Vol. 5, No. 5, October 1999, pp. 411-427. \\


\smallskip \\[-3pt]
\>    T. Hodes, R. H. Katz \\
\>\>      ``A Document-based Framework for Internet Application Control'' \\
\>\>       2nd USENIX Symposium on Internet Technologies and Systems \\
\>\>       Boulder, CO, October 1999, pp. 59-70. \\
\smallskip \\[-3pt]
\>    T. Hodes, J. Hauser, A. Freed, J. Wawrzynek\\
\>\>      ``Second-order Recursive Oscillators for Musical Additive Synthesis \\
\>\>\>        Applications on SIMD and VLIW Processors,'' \\
\>\>       International Computer Music Conference (ICMC), Beijing, China, October 1999. \\


\smallskip \\[-3pt]
\>    S. Czerwinski, B. Zhao, T. Hodes, A. Joseph, R. H. Katz \\
\>\>      ``An Architecture for a Secure Service Discovery Service'' \\
\>\>       5th ACM/IEEE International Conference on Mobile Computing \\
\>\>       Seattle, WA, August 1999, pp. 24-35. \\
\smallskip \\[-3pt]
\>    Steven McCanne, Eric Brewer, Randy Katz, Elan Amir, Yatin Chawathe,
    Todd Hodes, et. al. \\
\>\>      ``MASH: Enabling Scalable Multipoint Collaboration'' \\
\>\>       ACM Computing Surveys, Volume 31, No. 2es, June 1999. \\
\smallskip \\[-3pt]
\>    T. Hodes, J. Hauser, A. Freed, J. Wawrzynek, D. Wessel \\
\>\>      ``A Fixed-point Recursive Digital Oscillator for Additive Synthesis
        of Audio,'' \\
\>\>       IEEE International Conference on Acoustics, Speech, and Signal
            Processing,  \\
\>\>        Phoenix, Arizona, March 1999. \\
\smallskip \\[-3pt]
\>    T. Hodes, M. Newman, S. McCanne, R. H. Katz, J. Landay \\
\>\>      ``Shared Remote Control of a Videoconferencing Application: \\
\>\>\>            Motivation, Design, and Implementation,'' \\
\>\>       SPIE Multimedia Computing and Networking 1999,  \\
\>\>        San Jose, California, January 1999, pp. 17-28. \\
\smallskip \\[-3pt]

\end{tabbing}
\newpage
\begin{bf} \large
Patents \& Publications, cont.
\end{bf}
\begin{tabbing}
xxi \= xx \= x \= \kill

\>    E. Brewer, R. H. Katz, E. Amir, H. Balakrishnan, Y. Chawathe, A. Fox,
    S. Gribble, T. Hodes, G. Nguyen, \\
\>\>\> V. Padmanabhan, M. Stemm, S. Seshan, T. Henderson \\
\>\>      ``A Network Architecture for Heterogeneous Mobile Computing,'' \\
\>\>       IEEE Personal Communications Magazine, October 1998. Invited Paper. \\
\smallskip \\[-3pt]
\>    T. D. Hodes, R. H. Katz \\
\>\>      ``Enabling `Smart Spaces:' Entity Description and User Interface
        Generation \\
\>\>\>     for a Heterogeneous Component-Based Distributed System,'' \\
\>\>       DARPA/NIST Smart Spaces Workshop,  \\
\>\>       Gaithersburg, Maryland, July 1998.  pp. 7/44-7/51.  \\
\>\>       also, UC Berkeley Technical Report CSD/98/1008. \\


\smallskip \\[-3pt]
\>    T. D. Hodes, R. H. Katz, E. Servan-Schreiber, L. A. Rowe \\
\>\>      ``Composable Ad hoc Mobile Services for Universal Interaction,'' \\
\>\>       3rd ACM/IEEE International Conference on Mobile Computing,  \\
\>\>        Budapest, Hungary, September 1997, pp. 1-12. \\
\>\>        {\em Best Paper award. }\\
\smallskip \\[-3pt]
\>    T. D. Hodes \\
\>\>      ``Recursive Oscillators on a Fixed-Point Vector Microprocessor \\
\>\>\>     for High Performance Additive Synthesis of Audio,'' \\
\>\>       MS report, December 1997. also, UCB Technical Report CSD/98/1007. \\

\smallskip \\[-3pt]
\>    R. H. Katz, E. A. Brewer, E. Amir, H. Balakrishnan, A. Fox,
       S. Gribble, T. Hodes, \\
\>\>\>   D. Jiang, G. Nguyen, V. Padmanabhan, M. Stemm. \\
\>\> ``The Bay Area Research Wireless Access Network (BARWAN),''  \\
\>\> 41st IEEE Computer Society International Conference (COMPCON), 1996. \\



\smallskip \\
\>	W. T. Fennell, Jr., T. Hodes, S. Witherell, C. Goebel, 
	R. Thakkar, T. Schwenk, \\
\>\>	``Method of Managing Multi-Player Game Playing Over a Network,''  \\
\>\>	Patent \#5695400: BoxerJAM Films, Charlottesville, Virginia, accepted 12/97.  \\
\smallskip \\
\>    T. D. Hodes, B. A. McCoy, G. Robins \\
\>\>  ``Dynamically Wiresized Elmore-Based Routing Constructions,''  \\
\>\>   1994 IEEE International Symposium on Circuits and Systems, \\
\>\>   London, England, May 1994, Volume I, pp. 463-466. \\
\end{tabbing}

%\newpage
%\begin{bf} \large
%Publications, cont.
%%\end{bf}
%\begin{tabbing}
%xxi \= xx \= x \= \kill
%\end{tabbing}



\end{document}



