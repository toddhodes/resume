\documentstyle[palatino,url]{article}
\setlength{\oddsidemargin}{0.25 in}
\setlength{\evensidemargin}{-0.25 in}
%%\setlength{\topmargin}{-0.6 in}
\setlength{\topmargin}{-0.95 in}
\setlength{\textwidth}{6.5 in}
\setlength{\textheight}{9.6 in}
%%\setlength{\headsep}{0.55 in}
\setlength{\headsep}{0.35 in}
\setlength{\parindent}{0 in}
%%\setlength{\parskip}{0.1 in}
\setlength{\parskip}{0.05 in}


\addtolength{\evensidemargin}{-15pt}
\addtolength{\oddsidemargin}{-15pt}

\pagestyle{empty}

\begin{document}
\begin{center} \bf
\Large
                               Todd D. Hodes
\medskip
\normalsize


\begin{tabular}{lp{3.0in}r}
Richmond, CA 94805    & & todd@toddh.org \\
(510)  332-0399& & \url{toddhodes.com} \\
\end{tabular}

\end{center}

\bigskip


\begin{bf} \large
Education \\[-18pt]
\end{bf}

\begin{tabbing}
xxi \= \kill
\>   Ph.D. in Computer Science (Operating Systems - Networking), 2002  \\
\>   M.S. in Computer Science (Computer Architecture - Computer Music), 1997  \\
\>   University of California, Berkeley  \\
\smallskip \\[-6pt]
\>   B.S. in Computer Science, 1994 \\
\>   University of Virginia, Charlottesville \\
\end{tabbing}


\begin{bf} \large
Technical experience  \\[-16pt]
\end{bf}

\begin{itemize}
\item Kotlin, Java, shell (sh/bash/ksh), SQL, C, C++, C\#, JavaScript, python, Go, Rust. \\[-18pt]
\item Reactive programming: ReactiveX, RxKotlin, RxJava2, Java8 streams, Kotlin Flows. \\[-18pt]
\item Databases: MySQL, postgres, Realm, Room.  \\[-18pt]
\item AWS: EC2, S3, ELB, EMR, certs, security, provisioning.  \\[-18pt]
\item Containers: docker, kubernetes.  \\[-18pt]
\item Dependency injection: Dagger, Spring, Koin, Hilt.   \\[-18pt]
\item Android libraries:
	Kotlin, RxJava2, Mosby Model/View/Presenter (MVP), ViewModel, Jetpack, Material Design,
	   Amplitude, Adjust, Conductor, Branch.io, Cloudinary, Dagger, Epoxy, Lottie, EventBus, Crashlytics,
	   Google and MapQuest/MapBox Maps, Gson, Guava, Joda, OkHttp/Retrofit, Optimizely, Picasso, PubNub,
	   Realm, Room, Loggly. \\[-18pt]
\item Build tools: git, TeamCity, Jenkins, Tinderbox.   \\[-18pt]
\item Collaboration tools: Jira, Confluence, Miro, Zeplin, Bugzilla.   \\
\end{itemize}

%%%%%%%%%%%%%%%%%%%%%%%%%%%%%%%%

\begin{bf} \large
Work Experience
\end{bf}
\medskip

\begin{tabular}{p{5.2in}@{\hspace{1.5cm}}l}

	\bf  Avast, s.r.o. 	    & July 2016 -  \\
	\bf  Senior Staff Engineer	& January 2021 \\[4pt]
	
	   Avast acquired AVG in 2016 and took them private.  I moved to a senior staff engineer position,
	   reporting directly to the VP of Engineering.  In addition to coding, I ran a bi-weekly developer demo
	   and showcase series.
	   Continued working on the family of white-label family safety applications for cellular
	   telephone carriers, largely on the Verizon and AT\&T products as they were the highest revenue (most important).
	   Major contributor to the 1M+ LOC Android monorepo with modules per feature and flavors for each app,
	   an overall design I helped architect.
	   Provided programming \& architecture, 
		ownership of the entire CI/CD build lifecycle, feature ownership, tools support, team 
		technical leadership. Solved ongoing ownership issues around support tooling by writing and maintaining 
		scripts to handle what no one else was willing to, in addition to all the above. 
		
		 \\[9pt]
	   
	\bf AVG Technologies, LLC 	& October 2014 -  \\
	\bf  Director of Engineering (Mobile)  \& Architect 	& July 2016 \\[4pt]
	
	   AVG acquired LocationLabs in 2014 for \$220MM, an extremely successful exit.   I 
	   continued working on the android version of the white-label family safety applications. 
	   Worked in a github enterprise hosted monorepo with modules per feature and flavors for each app.\\[9pt]
	   
	\bf LocationLabs, LLC   & June 2010 -  \\
	\bf   Senior Software Engineer \& Director of Mobile Technologies 	& October 2014 \\[4pt]
	
	   In 2010, WaveMarket rebranded to LocationLabs in preparation for IPO or sale.
	   I continued working on the white-label family safety android applications. \\[9pt]
	   

\end{tabular}

\newpage
\begin{bf} \large
Work Experience, cont.
\end{bf}
\medskip

\begin{tabular}{p{5.2in}@{\hspace{1.5cm}}l}
	  
	\bf  WaveMarket, Inc.    & June 2002 -  \\
	\bf  Senior Scientist 	& June 2010 \\[4pt]
	
	   As one of the first employees at Berkeley-based WaveMarket/LocationLabs,
	   an early-stage startup that received its series A in 2001, 
	   we went from no deployed products in 2002 to an
	   acquisition by AVG Technologies in 2014.  I wore every engineering hat, starting with complete responsibility
	   for SysOps, server, and client -- writing tools and software for all three -- to
	   then specializing in Mobile Dev as we grew from a dozen engineers, added sysadmin, grew 
	   to thirty engineers, and finally to one hundred.  
	   Once iOS and Android unseated BlackBerry/J2ME/Symbian/PocketPC,
	   I specialized in Android, and, co-design of mobile architecture.  In addition to mobile dev, I provided
	   Partner Engineering/integration, visiting carriers around the world after we had signed contracts
	   with them and were hammering out integration points (authentication, location access, 
	   providing unmetered data atop metered plans, APIs/schema design, etc).
	   For details, please refer to the products highlighted in the {\bf Projects} section. \\[9pt]
       	  
    \bf Luxxon Corporation, San Jose, California. & 2000-2001 \\[3pt]
       Assisted the R\&D team, specifically
       protocols for discovery and negotiation of client
       device characteristics and media caching, in service of their media transcoding
       software and hardware. \\[9pt]

    \bf IBM Almaden Research Center, 
          Almaden, California                         & 1999-2000 \\[3pt]
       Graduate Internship that lead to follow-up work with
       the IBM Almaden T-Spaces middleware project, as detailed in the [IECON99] paper. \\[9pt]


\end{tabular}


%%%%%%%%%%%%%%%%%%%%%%%%%%%%

%\newpage
\medskip
\bigskip


\begin{bf} \large
Projects
\end{bf}
	   	   
	   \begin{itemize}
		\item 
		{\em Verizon Smart Family,
		AT\&T Secure Family, 
		T-Mobile FamilyWhere,
		Sprint Family Locator,
		Verizon FamilyBase,
		AT\&T FamilyMap} Android apps: 
		Major contributor to 1MM+ LOC Android monorepo; programming \& architecture, 
		feature ownership, API design, lead on Location and Controls 
		cross-functional horizontals, build process, infrastructure, testing, debug support, backward-compat,
		technical leadership, and related support tooling \& scripting. \\[-16pt]
		
		\item {\em  AT\&T FamilyMap} WindowsPhone app: Sole contributor, wrote our 
			implementation. AT\&T requested that we support it natively rather than as web-only. 
			Worked at the Microsoft campus in Redmond under NDA and security protocols
			finalizing the builds for when the OS was released.
			They loved it, app was preloaded on every Windows phone sold by AT\&T. \\[-16pt]

		\item \url{veriplace.com}: Carrier and non-carrier cell phone location as a API service. 
		A suite of thin smartphone application agents providing
		end-to-end user-plane locates (hybrid cell-sector/WiFi/GPS), with the option of using the 
		carrier networks (Verizon, AT\&T, Sprint) LPS/MLP/GMLC, or, avoiding carrier infrastructure for cost reasons.
		API/architecture, sole contributor to Windows Mobile implementation, contributor to Android/J2ME/RIM
		shared codebase.

		\item \url{streethive.com}: location-based social sharing site: StreetHive and Crunkie were social networks 
		focused on users sharing photos and text/comments atop a map. Supported on 
		handset clients (J2ME, RIM, xHTML, WAP) and web.  
		I worked on the handset client team, and shared responsibility for architecting the backend/APIs such that
		they were amenable to both web and 
		mobile content access. Java/J2ME front-end, Java/MySQL backend.

		\item \url{m.ask.com}: One of the worlds first mobile portals.  xHTML-MP/WAP access to the ask.com
		syndication backend. Provided access to full web search, 
		image search, maps with turn-by-turn directions, weather, horoscopes, and more. 
		Implemented with Java servlets/JSP + MySQL 
		by myself and one other engineer.  {\em The site won two Webby Awards in the Mobile category 
		(both Jury and People's Choice).}

		\item \url{cmprssr.com}: In order to view the resulting content discovered via queries on m.ask.com, web 
		pages needed to be transcoded from free-form HTML to valid xHTML-MP. cmprssr provided this service at
		extremely high volume (millions/day). Written in perl by myself and one other engineer.

  		\item IAC \url{gps.ask.com}: voice turn-by-turn navigation on GPS-enabled feature phones
		 to aggregated InterAactiveCorp (IAC) properties CitySearch, Evite, Ticketmaster, Ask.com. The underlying technology innovation
		  used to manage app complexity under memory and processing constraints was a declarative UI screen specification
		   augmented with novel XML-defined action flows called “chains”. These XML descriptions were compiled down
		    into primitives for efficiency, and could be updated at runtime, to overcome the lack of a ClassLoader 
		    in JavaME. 
	   
  		\item {\em MapMe} \& {\em NearHere}:
		on-handset point-of-interest finders + route generator, communicating to a custom geoserver
		backend. WaveMarket’s first handset software product releases. Responsible for hardware acquisition
		and co-location hosting,
		OS installation and configuration, sysadmin / operations, Cisco VPN to partner site, hardware load-balancing
		and failover (via Alteon ACEdirector), data integration, plus all the client and server application software itself.
		JAR code size was limited to 100KB, heap to 256KB.
		
	\end{itemize}

%%%%%%%%%%%%%%%%%%%%%%%%

\bigskip
\begin{bf} \large
Patents \& Publications
\end{bf}
\begin{tabbing}
xxi \= xx \= x \= \kill

\>    Jesse Meyers, Scott Hotes, Todd Hodes \\
\>\>      ``System and method for range search over distributive storage systems'' \\
\>\>    Patent \#8924365, issued 12/2014. \\

\smallskip \\[-3pt]
\>    Brian Martin, Joseph Augst, Jesse Meyers, Todd Hodes, Scott Hotes \\
\>\>      ``System and method for managing third party application program access to user information \\
\>\>      via a native application program interface'' \\
\>\>    Patent \#8683554, issued 3/2014. \\

\smallskip \\[-3pt]
\>    Adrian Freed, Todd Hodes, John Hauser \\
\>\>      ``Apparatus and Method of Additive Synthesis of Digital Audio Signals \\
\>\>   Using a Recursive Digital Oscillator'' \\
\>\>    Patent \#7317958: UC Berkeley Regents, Berkeley, California, issued 1/2008. \\

\smallskip \\[-3pt]
\>    Todd Hodes \\
\>\>      ``Discovery and Adaptation for Location-Based Services'' \\
\>\>    Ph.D. thesis, University of Califonia, Berkeley. 2002. \\
\>\>    Randy Katz, thesis advisor \\

\smallskip \\[-3pt]
\>    T. Hodes, S. Czerwinski, B. Zhao, A. Joseph, R. H. Katz \\
\>\>      ``An Architecture for Secure Wide-area Service Discovery'' \\
\>\>       ACM Wireless Networks Journal, Special Issue \\
\>\>       Volume 8, Issue 2/3, March/May 2002, pp. 213-230 \\

\smallskip \\[-3pt]
\>   M. Munson, T. Hodes, T. Fischer, K. H. Lee, T. Lehman, B. Zhao \\
\>\>      ``Flexible Internetworking of Devices and Controls'' \\
\>\>        25th Annual Conference of the IEEE Industrial Electronics Society (IECON),  \\
\>\>	    San Jose, CA, December 1999. \\

\smallskip \\[-3pt]
\>    T. D. Hodes, R. H. Katz \\
\>\>      ``Composable Ad hoc Location-based Services for Heterogeneous
             Mobile Clients,'' \\
\>\>       ACM Wireless Networks Journal, Special issue on Mobile Computing \\
\>\>       Vol. 5, No. 5, October 1999, pp. 411-427. \\
\smallskip \\[-3pt]
\>    T. Hodes, R. H. Katz \\
\>\>      ``A Document-based Framework for Internet Application Control'' \\
\>\>       2nd USENIX Symposium on Internet Technologies and Systems \\
\>\>       Boulder, CO, October 1999, pp. 59-70. \\
\smallskip \\[-3pt]
\>    T. Hodes, J. Hauser, A. Freed, J. Wawrzynek\\
\>\>      ``Second-order Recursive Oscillators for Musical Additive Synthesis \\
\>\>\>        Applications on SIMD and VLIW Processors,'' \\
\>\>       International Computer Music Conference (ICMC), Beijing, China, October 1999. \\

\end{tabbing}
\newpage
\begin{bf} \large
Patents \& Publications, cont.
\end{bf}
\begin{tabbing}
xxi \= xx \= x \= \kill





\smallskip \\[-3pt]
\>    S. Czerwinski, B. Zhao, T. Hodes, A. Joseph, R. H. Katz \\
\>\>      ``An Architecture for a Secure Service Discovery Service'' \\
\>\>       5th ACM/IEEE International Conference on Mobile Computing \\
\>\>       Seattle, WA, August 1999, pp. 24-35. \\
\smallskip \\[-3pt]
\>    Steven McCanne, Eric Brewer, Randy Katz, Elan Amir, Yatin Chawathe,
    Todd Hodes, et. al. \\
\>\>      ``MASH: Enabling Scalable Multipoint Collaboration'' \\
\>\>       ACM Computing Surveys, Volume 31, No. 2es, June 1999. \\
\smallskip \\[-3pt]
\>    T. Hodes, J. Hauser, A. Freed, J. Wawrzynek, D. Wessel \\
\>\>      ``A Fixed-point Recursive Digital Oscillator for Additive Synthesis
        of Audio,'' \\
\>\>       IEEE International Conference on Acoustics, Speech, and Signal
            Processing,  \\
\>\>        Phoenix, Arizona, March 1999. \\
\smallskip \\[-3pt]
\>    T. Hodes, M. Newman, S. McCanne, R. H. Katz, J. Landay \\
\>\>      ``Shared Remote Control of a Videoconferencing Application: \\
\>\>\>            Motivation, Design, and Implementation,'' \\
\>\>       SPIE Multimedia Computing and Networking 1999,  \\
\>\>        San Jose, California, January 1999, pp. 17-28. \\
\smallskip \\[-3pt]
\>    E. Brewer, R. H. Katz, E. Amir, H. Balakrishnan, Y. Chawathe, A. Fox,
    S. Gribble, T. Hodes, G. Nguyen, \\
\>\>\> V. Padmanabhan, M. Stemm, S. Seshan, T. Henderson \\
\>\>      ``A Network Architecture for Heterogeneous Mobile Computing,'' \\
\>\>       IEEE Personal Communications Magazine, October 1998. Invited Paper. \\
\smallskip \\[-3pt]
\>    T. D. Hodes, R. H. Katz \\
\>\>      ``Enabling `Smart Spaces:' Entity Description and User Interface
        Generation \\
\>\>\>     for a Heterogeneous Component-Based Distributed System,'' \\
\>\>       DARPA/NIST Smart Spaces Workshop,  \\
\>\>       Gaithersburg, Maryland, July 1998.  pp. 7/44-7/51.  \\
\>\>       also, UC Berkeley Technical Report CSD/98/1008. \\


\smallskip \\[-3pt]
\>    T. D. Hodes, R. H. Katz, E. Servan-Schreiber, L. A. Rowe \\
\>\>      ``Composable Ad hoc Mobile Services for Universal Interaction,'' \\
\>\>       3rd ACM/IEEE International Conference on Mobile Computing,  \\
\>\>        Budapest, Hungary, September 1997, pp. 1-12. \\
\>\>        {\em Best Paper award. }\\
\smallskip \\[-3pt]
\>    T. D. Hodes \\
\>\>      ``Recursive Oscillators on a Fixed-Point Vector Microprocessor \\
\>\>\>     for High Performance Additive Synthesis of Audio,'' \\
\>\>       MS report, December 1997. also, UCB Technical Report CSD/98/1007. \\

\smallskip \\[-3pt]
\>    R. H. Katz, E. A. Brewer, E. Amir, H. Balakrishnan, A. Fox,
       S. Gribble, T. Hodes, \\
\>\>\>   D. Jiang, G. Nguyen, V. Padmanabhan, M. Stemm. \\
\>\> ``The Bay Area Research Wireless Access Network (BARWAN),''  \\
\>\> 41st IEEE Computer Society International Conference (COMPCON), 1996. \\
\smallskip \\
\>	W. T. Fennell, Jr., T. Hodes, S. Witherell, C. Goebel, 
	R. Thakkar, T. Schwenk, \\
\>\>	``Method of Managing Multi-Player Game Playing Over a Network,''  \\
\>\>	Patent \#5695400: BoxerJAM Films, Charlottesville, Virginia, accepted 12/97.  \\
\smallskip \\
\>    T. D. Hodes, B. A. McCoy, G. Robins \\
\>\>  ``Dynamically Wiresized Elmore-Based Routing Constructions,''  \\
\>\>   1994 IEEE International Symposium on Circuits and Systems, \\
\>\>   London, England, May 1994, Volume I, pp. 463-466. \\
\end{tabbing}

%\newpage
%\begin{bf} \large
%Publications, cont.
%%\end{bf}
%\begin{tabbing}
%xxi \= xx \= x \= \kill
%\end{tabbing}



\end{document}



